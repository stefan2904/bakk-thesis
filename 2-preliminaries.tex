\chapter{Preliminaries}

\section{Symmetric Cryptography} \todo{more formal explanation needed?}

A encryption function (cipher) is a transformation in the sense that it transfers a given \textit{plaintext} so that its meaning cannot be understood anymore \citep[section 1.4]{HAC}. In the context of \textit{Symmetric Cryptography} this transfer is parametrized by a so called \textit{encryption key}  \citep[section 1.5]{HAC}.

\begin{equation} \label{eq:skencrypt}
	cipherText = encryptCipher(plainText, key)
\end{equation}

\begin{equation} \label{eq:skdecrypt}
	plainText = decryptCipher(cipherText, key)
\end{equation}


\section{Public-Key Cryptography}

In contrast to  {Symmetric Cryptography}, \textit{Piblic-Key Cryptography} references a class of cryptographic algorithms which require two different keys. \citep[section 1.8]{HAC} The so called public key is used to encrypt a message. On the other hand, a private key is needed to decrypt the message. The process principle is illustrated in equations \ref{eq:pkencrypt} and \ref{eq:pkdecrypt}. 

\begin{equation} \label{eq:pkencrypt}
	cipherText = encryptCipher(plainText, publicKey)
\end{equation}

\begin{equation} \label{eq:pkdecrypt}
	plainText = decryptCipher(cipherText, privateKey)
\end{equation}

%\section{Public Key Infrastructure \&{} X.509}
%\citep{RFC5280}

%\section{S/MIME}
%\citep{RFC5751}

\section{Digital Signature}


\section{Message Integrity Protection}

\section{SSL / TLS}

\textit{SSL} (Secure Sockets Layer) and \textit{TLS} (Transport Layer Security) are protocols which provide a way to create secure connections over computer networks like the Internet.  TLS ist the successor of SSL. A connection established using TLS is secure in a way that an attacker with access to the connection is not able to change transmitted data or to eavesdrop it. \cite{RFC5246} A TLS connection is established between two hosts and lasts only as long as the actual communication. The participants of the communication can be authenticated by a \textit{Public Key Infrastructure} (PKI) using \textit{X.509 certificates}.

\section{E-Mail}

The \textit{E-Mail} communication system consists of several protocols to transmit messages from one person to one or more other people over computer networks. Used protocols such as \textit{IMAP} (Internet Message Access Protocol), \textit{POP3} (Post Office Protocol) and \textit{SMTP} (Simple Mail Transfer Protocol) don't provide encryption and transmit data in plan from one host to another. Although there exist secure versions of the named protocols, the sender of a message is not able to assure that a message will be secured along the whole transmission. \todo{References for this claims?} 

\section{Java Crypto Architecture} \label{section:pre:jca}

% TODO !

\section{IAIK-JCE} \label{section:pre:jce}

% TODO !




