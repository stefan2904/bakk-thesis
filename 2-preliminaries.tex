\chapter{Preliminaries}
\label{chapter:pre}

In this chapter we explain the basic concepts required to understand this thesis. We explain the difference between symmetric and asymmetric cryptography. In addition, we describe the meaning of key ID, fingerprint and keyring. Furthermore, we show the principles of cryptography in Java and the library we used for our implementation.

\section{Symmetric Cryptography} \label{section:pre:symcrypto}


An encryption function (cipher) is a transformation which transfers a given \textit{plaintext} so that its meaning cannot be understood anymore \citep[section 1.4]{HAC}. In the context of \textit{symmetric cryptography} this transfer is parametrized by a so called \textit{encryption key}  \citep[section 1.5]{HAC}. Equations \ref{eq:skencrypt} and \ref{eq:skdecrypt} illustrate this process.

\begin{equation} \label{eq:skencrypt}
	\text{cipherText} = \text{encryptCipher}(\text{plainText}, \text{key})
\end{equation}

\begin{equation} \label{eq:skdecrypt}
	\text{plainText} = \text{decryptCipher}(\text{cipherText}, \text{key})
\end{equation}

% FEEDBACK::::::::::::::::::::::::::::::::::::::::::::::::::::::::::::::::::::::
% Something about MACs / Authenticated Encryption?
% /FEEDBACK:::::::::::::::::::::::::::::::::::::::::::::::::::::::::::::::::::::

\section{Public-Key Cryptography}
\label{section:pre:publiccrypto}

In contrast to  {symmetric cryptography}, \textit{public-key cryptography} references a class of cryptographic algorithms which require two different keys \citep[section 1.8]{HAC}. The so called public key is used to encrypt a message. On the other hand, a private key is needed to decrypt the message. Therefore, public-key cryptography is also called \textit{asymmetric cryptography}.  The process principle is illustrated in equations \ref{eq:pkencrypt} and \ref{eq:pkdecrypt}. 

\begin{equation} \label{eq:pkencrypt}
	\text{cipherText} = \text{encryptCipher}(\text{plainText}, \text{publicKey})
\end{equation}

\begin{equation} \label{eq:pkdecrypt}
	\text{plainText} = \text{decryptCipher}(\text{cipherText}, \text{privateKey})
\end{equation}

% FEEDBACK::::::::::::::::::::::::::::::::::::::::::::::::::::::::::::::::::::::
% Something about signing?
% /FEEDBACK:::::::::::::::::::::::::::::::::::::::::::::::::::::::::::::::::::::

%\section{Public Key Infrastructure \&{} X.509}
%\citep{RFC5280}

%\section{S/MIME}
%\citep{RFC5751}

%\section{Digital Signature}


\section{Key ID \&{} Fingerprint}
\label{section:pre:keyid}

A fingerprint is a character or byte sequence which is used to identify a key.
A \textit{key ID} is the short form of a fingerprint.
For example the fingerprint can be the 160-bit \myacro{SHA-1} hash of a key, while the key id are the lower 64 bits of the fingerprint. Due to the nature of a hash function it is possible that collisions exist.
This is further explained in \cite[section 12.2]{RFC4880}.

%\section{Message Integrity Protection}

\section{Keyring}
\label{section:pre:keyring}

A \textit{keyring} is a persistent collection of keys. 
In general a keyring is a list of public- and private-keys. It can be saved to a disc or stored otherwise. 
A keyring stores all the keys used by a cryptographic software.
This is further explained in \cite[section 3.6]{RFC4880}.

%\section{SSL / TLS}

%\textit{SSL} (Secure Sockets Layer) and \textit{TLS} (Transport Layer Security) are protocols which provide a way to create secure connections over computer networks like the Internet.  TLS ist the successor of SSL. A connection established using TLS is secure in a way that an attacker with access to the connection is not able to change transmitted data or to eavesdrop it. \cite{RFC5246} A TLS connection is established between two hosts and lasts only as long as the actual communication. The participants of the communication can be authenticated by a \textit{Public Key Infrastructure} (PKI) using \textit{X.509 certificates}.

%\section{E-Mail}

%The \textit{E-Mail} communication system consists of several protocols to transmit messages from one person to one or more other people over computer networks. Used protocols such as \textit{IMAP} (Internet Message Access Protocol), \textit{POP3} (Post Office Protocol) and \textit{SMTP} (Simple Mail Transfer Protocol) don't provide encryption and transmit data in plan from one host to another. Although there exist secure versions of the named protocols, the sender of a message is not able to assure that a message will be secured along the whole transmission. \todo{References for this claims?} 

\section{Java Crypto Architecture} \label{section:pre:jca}

%The \textit{Java Crypto Architecture} (\myacro{JCA}) provides a basic framework for working with cryptographic primitives in Java \cite{JCA}. Furthermore it defines an API for working with encryption and other more advanced cryptographic primitives.
The \textit{Java Crypto Architecture} (\myacro{JCA}) provides a basic framework, including an API, for working with cryptographic primitives in Java \cite{JCA}.
%The \textit{Java Cryptography Extension} (\myacro{JCE}) provides implementation of cryptographic algorithms. \myacro{JCE} implements the API defined by \myacro{JCA}. 
%A \textit{Java Cryptography Extension} (\myacro{JCE}) provides implementation of cryptographic algorithms, and conforms to the API defined by \myacro{JCA}.

\section{IAIK-JCE} \label{section:pre:jce} 

% FEEDBACK::::::::::::::::::::::::::::::::::::::::::::::::::::::::::::::::::::::
% Fix explenation of JCE; this definition is hystorical
% /FEEDBACK:::::::::::::::::::::::::::::::::::::::::::::::::::::::::::::::::::::

\myacro{IAIK-JCE} is the \textit{Java Cryptography Extension} (\myacro{JCE})  implementation of the \textit{Institute of Applied Information Processing and Communications} (\myacro{IAIK}). It is ``a set of APIs and implementations of cryptographic functionality, including hash functions, message authentication codes, symmetric, asymmetric, stream, and block encryption, key, and certificate management'' \cite{IAIKJCE}.



