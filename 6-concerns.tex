\chapter{Conclusion and Concerns}

Here (and not in Chapter 4) do you describe how well your idea
worked. Talk about the experimental setup. Give the results. If
applicable, give tables and point out general trends.

Conclusions (evtl eigenes Kapitel?)

Recapitulate the story in hindsight. State advantages and
disadvantages again.

\chapter{Concerns}

%\subsection{Security Evaluation Of Cryptographic Schemes}

The following chapter gives an overview of concerns with the openPGP standard.
A detailed security evaluation of openPGP was out of scope of this thesis. \\

It should be noted that at the time of writing of this thesis the used schemes were undergoing a security evaluation \citep{TUB2015}.

\section{Keylengths}

Various algorithms are used in openPGP, whose strengths are not strictly defined in the standard. 
Only a recommendation of hash sizes and key lengths is given \citep[section 14]{RFC4880}.
The recommended values at the time of writing of the openPGP standard are shown in table \ref{tab:keylengths}.

\begin{table}[h]
	\centering
	\begin{tabular}{|c|c|c|c|}
		\hline Date & 			Asymmetric & Hash & Symmetric \\ 
		\hline\hline 2010 (Legacy) & 1024 & 160 & 80 \\ 
		\hline 2011 - 2030 & 2048 & 224 & 112 \\ 
		\hline $>$ 2030 		   & 3072 & 256 & 128 \\ 
		\hline $>>$ 2030       & 7680 & 384 & 192 \\ 
		\hline $>>>$ 2030     & 15360 & 512 & 256 \\ 
		\hline 
	\end{tabular}
	\label{tab:keylengths}
	\caption{Strengths for algorithms as recommended in \citep[section 14]{RFC4880}} 
\end{table}

The current NIST and BSI recommendations for cipher and hash strengths confirm the recommendations given \cite{keylenNIST} \cite{KeylenBSI}. 

\section{Algorithms}

% MD5: deprecated

% SHA-1: modification detection nur mit sha1 
%             http://tools.ietf.org/html/rfc4880#section-5.13
%             http://tools.ietf.org/html/rfc4880#section-13.11

Most of the algorithms used by openPGP are still considered save. \todo{Referenz?}
The two exceptions are the \textit{MD5} and \textit{SHA-1} algorithms used for digital signatures and message authentication. 

\textit{MD5} is already deprecated in the current openPGP standard \citep[section 14]{RFC4880} and implementations are not allowed to create signatures using \textit{MD5}. 

OpenPGP is not using authenticated ciphers to detect modifications to encrypted messages.
Instead a own scheme is used, supporting only the \textit{SHA-1} hash algorithm \citep[section 5.13]{RFC4880}. 
Following the finding of sufficient serious attacks on \textit{SHA-1} \cite{SHA1attack} it should be considered to upgrade this scheme to a hash algorithm of the \textit{SHA-2} family or \textit{SHA-3}, or by using a mode of operation to achieve authenticated encryption. The openPGP standard makes some suggestions to allow such extensions \citep[section 13.11]{RFC4880} but fails to suggest authenticated ciphers or modes of operation.


\section{Sign-then-Encrypt}



\section{The Case On Usability}


% KEY EXCHANGE

Due to the nature of non-transparent encryption and asynchronous communication \citep[section 2.1]{RFC4880}, the sender's and receiver's openPGP implementations never directly talk to each other.
To establish a end to end encrypted communication it is therefor necessary to manually create the key-pairs and to establish a manual key exchange as describe in chapter XXX. \todo{ref to chapter with key exchange and web of trust}
This results in a number of security and usability problems. \\

% TRUST / WEB OF TRUST

% SIDECHANNEL: KeyServer knows whick keys you requested, MITM possible if HTTP and not HTTPS

% WEBMAIL

% SUBKEYS

% http://blog.cryptographyengineering.com/2014/08/whats-matter-with-pgp.html
% https://pthree.org/2014/08/18/whats-the-matter-with-pgp/
% http://thehackernews.com/2014/08/cryptography-expert-pgp-encryption-is_19.html




