\chapter{openPGP} \label{chapter:openpgp}

In this chapter we give an introduction into \textit{openPGP}. 
We present its history and explain the fundamental principles. 
Furthermore we list some common use-cases of openPGP implementations
and compare openPGP with related protocols.

% explain what it is, what it does and how it does it
% dont explain packet stuff in this chapter!

\section{History}

\section{Principles} \label{section:openpgp:principles}

% combination sym cryp o + public crypto
% key pairs
% (key id / fingerprint)
% (key exchange?)

\subsection{PGP Key-Pairs} \label{section:openpgp:keys}

\subsection{Message Confidentiality} \label{section:openpgp:confidentiality}


\subsection{Message Integrity} \label{section:openpgp:integrity}

% via signature and/or integrity protected packets

\subsection{Sender Authentication} \label{section:openpgp:authentication}


% explain key signing here

% explain key servers here? "Key Distribution"

\subsection{Key Distribution} \label{section:openpgp:keydistribution}

To encrypt a message one needs the openPGP public key of the designated receiver of the message.
To verify the signature of a received message, the openPGP public key of the sender is needed, respectively.
OpenPGP users therefore have to distribute their openPGP public keys.

The easiest way to do perform this \textbf{key-exchange} is by sending the openPGP public key along with every message or by publishing it on a website. 

Another way to distribute openPGP public keys is by publishing them to a designated key directory, a so called \textbf{key-server}. \\

Such an key-exchange via an untrusted channel results in the need to verify the authenticity and integrity of the received key, as described in sections \ref{section:openpgp:trustmodel} and \ref{section:concerns:usability}.

\subsection{Trust Model} \label{section:openpgp:trustmodel}

In the openPGP ecosystem it is in general not possible to transmit the openPGP public-key via an secure channel. Therefore it is necessary to validate the integrity and authenticity of a public-key before being able to use the key to validate the signature on a received message. It is necessary to validate this properties before encrypting to a public-key, respectively. \\

In the simplest case it is sufficient to hash the sensitive key-parts and then compare the hash-digest via an secure channel (for example via phone or by printing it on a business card). In openPGP this hash over the public-key is called a \textbf{fingerprint} \citep[section 12.2]{RFC4880}.

When it is not possible to compare this fingerprint, for example when the communication parties do not know each other in advance, a more sophisticated approach is needed. For such cases openPGP provides its own trust model, called the \textbf{Web of Trust}.

In contrast to hierarchical trust models, in the \textit{Web of Trust} it is possible for every key to certify every other key. \todo{what does "certify" mean?}
This results in a (directed) graph of trust. \\

% expail WoT here?

A detailed explanation of the \textit{Web of Trust} approach was out of scope of this thesis. A analysis of the openPGP \textit{Web of Trust} can be found in \cite{Ulrich2011}. \todo{move this to "related work"?}

\subsection{Key Revocation}







% maybe not / shorter (in one section without subsection?):

\section{Applications}

%\subsection{E-Mail and PGP/Mime}

%\subsection{TLS}

%\section{Implementations}

%\subsection{PGP}

%\subsection{GnuPG / Gpg4win}

%\subsection{BouncyCastle}

%\subsection{APG}

\section{Similar protocols ("Related work")}

The location of Chapter 3 is always a point of contention. It
could also be the last chapter before the conclusions, or it could
be a section in the introduction. You choose. In any case, you will
have to embed related work throughout your thesis to explain
the difference in details between your work and others and to
explain where you have copied ideas.

Again, related work is a chance to show that you have looked
around. You should ideally include any work that someone could
think makes your work obsolete. You should not include unrelated
work and for every paper you mention, you should explain the
difference to your work. Try to tell a story, rather than giving a
list.

Chapter 3 does not contain anything new. You can give some critique
of the related work, which is obviously your opinion, but in this
chapter, such a critique is usually not very long.

\subsection{S/MIME}