\chapter{openPGP}

% explain what it is, what it does and how it does it
% dont explain packet stuff in this chapter!

\section{History}

\section{Principles} \label{section:openpgp:principles}

% combination sym cryp o + public crypto
% key pairs
% (key id / fingerprint)
% (key exchange?)

\subsection{Message Confidentiality} \label{section:openpgp:confidentiality}


\subsection{Message Integrity} \label{section:openpgp:integrity}

% via signature and/or integrity protected packets

\subsection{Sender Authentication} \label{section:openpgp:authentication}


% expail WoT here?



% maybe not:

\section{Applications}

\subsection{E-Mail and PGP/Mime}

\subsection{TLS}

\section{Implementations}

\subsection{PGP}

\subsection{GnuPG / Gpg4win}

\subsection{BouncyCastle}

\subsection{APG}

\section{Similar protocols ("Related work")}

The location of Chapter 3 is always a point of contention. It
could also be the last chapter before the conclusions, or it could
be a section in the introduction. You choose. In any case, you will
have to embed related work throughout your thesis to explain
the difference in details between your work and others and to
explain where you have copied ideas.

Again, related work is a chance to show that you have looked
around. You should ideally include any work that someone could
think makes your work obsolete. You should not include unrelated
work and for every paper you mention, you should explain the
difference to your work. Try to tell a story, rather than giving a
list.

Chapter 3 does not contain anything new. You can give some critique
of the related work, which is obviously your opinion, but in this
chapter, such a critique is usually not very long.

\subsection{S/MIME}