\chapter{Introduction}

\todo{References!?}

It is \textit{Year 1 after Snowden} and we lost trust in every single component of digital communication. In every single one? No! A tiny standard resists the pressure and keeps our secrets what they should be, secret. I proudly introduce: openPGP!


The introduction is your entire thesis in, say, 1-2 pages. If someone reads
the intro, she should have understood your thesis. Maybe not all the
details, but nothing that comes afterwards will be a surprise.

This chapter is the second thing you write. Needs a major overhaul
at the end. Why do you write Abstract and Introduction first?
Because it forces you to think about the structure of the work and
the reasons for doing what you did. There is nothing worse than
writing four chapters and then figuring out you did not understand
the rationale for your work --- you would have to rewrite a lot.

The structure of the introduction is as follows:

1. Background. What are you talking about? A new methods for public
key cryptography? Then talk about communication security. A new
algorithm for image recognition? Talk about the importance of image
recognition is modern life.

2. The problem. Something was wrong in the world when you started
writing your thesis. What was it? Why was it a problem? Here you
state the problem and set concrete goals for a solution. For
instance, people with quantum computers could break everyone's
communication. OR we have a new algorithm, but we are not sure if
it is secure. This needs to be solved to make the world a better
place.

3. Your solution. How did you address the problem? Describe what
you did, and in how far it solves the problem. Describe drawbacks
as well, but they should be secondary. Your contribution could be
that now we know that the given algorithm is insecure. This being
a Master's Thesis, you are not expected to fulfill all the goals you
set.

4.Structure of thesis. What are you going to describe in which
chapter?

Again, the introduction is important. By the end of it, you have stated
that your field of research is important, that there was a problem and
that you have solved some aspect of it. In some sense, it is a sales
pitch,
but a sales pitch that does not exagerate anything.

Throughout this chapter and the other chapters, you will sprinkle
references to related work to show that you know what is going on
beside your work. You use references to back up claims
("X.509 is generally considered to be faulty
[ImportantResearcher13]"), but also to give the other side of the
debate without spending too many words ("although the importance is
privacy in the facebook age is under discussion [Zuc12]").

Under no circumstance do you claim something without backing it up.
If you have a wild personal claim that you cannot back up ("SSL is
broken") you use it as a working assumption ("In this thesis, we
will examine alternatives to the SSL architecture that are immune
to the possible criticism that ... .") A favorite trick of the
trade is to downplay a claim. For instance, instead of saying that
something is insecure, you can say that its insecurity has been
argued by someone, or even that security is not completely obvious.
The reader understands what you are really saying, and for practical
purposes, these claims mean about the same. (But don't underclaim
either!)