%%%% Time-stamp: <2013-02-25 10:31:01 vk>


\chapter*{Abstract}
\label{cha:abstract}


\emph{OpenPGP} is an Internet standard for securely sending messages over insecure networks like the Internet. Providing end-to-end encryption by combining asymmetric and symmetric cryptography, trust in any network component except the sender's and receiver's computer is not needed. Furthermore it guarantees for the integrity of messages using digital signatures. openPGP also provides a system to verify the identity of participants of an communication using a trust model called the Web of Trust. \\

 % TODO: packets hier schon erwähnen?

In this thesis we give an overview of the principles of openPGP and the underlying Internet standard, the\textit{ openPGP Message Format}. 
Additionally we show in which ways openPGP implementations are used.
We explain the functionality and inner workings. Furthermore we show an example on the basis of our implementation of the openPGP standard in Java using the \emph{IAIK-JCE} cryptographic library.
%We also provide a comparison of other openPGP implementations and alternatives.
Besides that we discuss some considerations regarding security and usability. 
We do not describe the structures and algorithms in detail, since this is already covered by the standard's document. \\
\\

%  and make suggestions for a upcoming version of the openPGP standard.

% TODO: ergebnisse!?

\textbf{Keywords:} \mykeywords

%\glsresetall %% all glossary entries should be used in long form (again)
%% vim:foldmethod=expr
%% vim:fde=getline(v\:lnum)=~'^%%%%\ .\\+'?'>1'\:'='
%%% Local Variables:
%%% mode: latex
%%% mode: auto-fill
%%% mode: flyspell
%%% eval: (ispell-change-dictionary "en_US")
%%% TeX-master: "main"
%%% End:
