%%%% Time-stamp: <2013-02-25 10:31:01 vk>


\chapter*{Abstract}
\label{cha:abstract}

\todo{TODO}


\emph{OpenPGP} is a standard for securely sending messages over insecure networks like the Internet. Providing end-to-end encryption by combining asymmetric and symmetric cryptography, trust in any network component except the sender's and receiver's computer is not needed. Furthermore it guarantees for the integrity of messages using digital signatures. openPGP also provides a system to verify the identity of participants of an communication using signed keys. \\
 
 % TODO: packets hier schon erwähnen?
 
 In this thesis we explain the principles of the so called openPGP message format. 
 We explain the inner workings and show an example on the basis of our implementation 
 of the openPGP standard in Java using the \emph{IAIK-JCE} cryptographic library. 
 Additionally, we show in which way openPGP implementations are, or could be, used.
 %We also provide a comparison of other openPGP implementations and alternatives. 
 Besides that we discuss some considerations regarding security and usability. \\

%  and make suggestions for a upcoming version of the openPGP standard.

% TODO: ergebnisse!?




\textbf{Keywords:} \mykeywords

%\glsresetall %% all glossary entries should be used in long form (again)
%% vim:foldmethod=expr
%% vim:fde=getline(v\:lnum)=~'^%%%%\ .\\+'?'>1'\:'='
%%% Local Variables:
%%% mode: latex
%%% mode: auto-fill
%%% mode: flyspell
%%% eval: (ispell-change-dictionary "en_US")
%%% TeX-master: "main"
%%% End:
