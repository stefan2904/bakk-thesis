%%%% Time-stamp: <2013-02-25 10:31:01 vk>


\chapter*{Abstract}
\label{cha:abstract}


\myacro{OpenPGP} is an Internet standard for securely sending messages over insecure networks like the Internet. It provides end-to-end encryption by combining asymmetric and symmetric cryptography. Trust in any network component except the sender's and receiver's computer is not needed. Furthermore, it guarantees for the integrity of messages using digital signatures. OpenPGP also provides a system for verification of the identity of participants of an communication using a trust model called the Web of Trust. \\

 % TODO: packets hier schon erwähnen?

In this thesis we give an overview of the principles of \myacro{OpenPGP} and the underlying Internet standard, the \textit{OpenPGP Message Format}. 
Additionally, we show in which ways OpenPGP implementations are used.

We explain the functionality and inner workings of OpenPGP. Furthermore we show an example on the basis of our implementation of the OpenPGP standard in Java using the \myacro{IAIK-JCE} cryptographic library.
%We also provide a comparison of other OpenPGP implementations and alternatives.
We also discuss considerations regarding security and usability.
%We do not describe the structures and algorithms in detail, since this is already covered by the standard's document. \\
\\

%  and make suggestions for a upcoming version of the OpenPGP standard.

% TODO: ergebnisse!?

\textbf{Keywords:} \mykeywords

%\glsresetall %% all glossary entries should be used in long form (again)
%% vim:foldmethod=expr
%% vim:fde=getline(v\:lnum)=~'^%%%%\ .\\+'?'>1'\:'='
%%% Local Variables:
%%% mode: latex
%%% mode: auto-fill
%%% mode: flyspell
%%% eval: (ispell-change-dictionary "en_US")
%%% TeX-master: "main"
%%% End:
