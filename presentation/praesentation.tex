% LaTeX Präsentationsvorlage (2013) der TU Graz, rev12, 2013/01/31
\documentclass{beamer}
% \documentclass[aspectratio=169]{beamer}
% \usetheme{tugraz2013}
% \usetheme[notes]{tugraz2013}
\usetheme[minimal]{tugraz2013}

%% Titelblatt-Einstellungen
\title[openPGP in Java]{An implementation\\ of the OpenPGP Message Format\\ in Java}
\author{Stefan More}
% \date{Graz, XX. Dezember 2010}		% \today für heutiges Datum verwenden
\date{June 13, 2014}
\institute[Bakk@IAIK]{\\Advisor: Dieter Bratko}
%\instituteurl{iaik.tugraz.at}
\institutelogo{javaSec.jpg}
\additionallogo{institutslogo.png}

%%%%%%%%%%%%%%%%%%%%%%%%%%%%%%%%%%%%%%%%%%%%%%%%%%%%%%%%%%%%%%%%%%%%%%%%%%%%
\begin{document}
%%%%%%%%%%%%%%%%%%%%%%%%%%%%%%%%%%%%%%%%%%%%%%%%%%%%%%%%%%%%%%%%%%%%%%%%%%%%
\titleframe

\begin{frame}
  \frametitle{Agenda}
  \tableofcontents%[hideallsubsections] 
  \note{
  	I'm going to talk about \ldots
  }
\end{frame}


\section{Agenda}
\begin{frame}
	\frametitle{Ziele der ersten Vorlesung}
	Einführung in die KORE
	\begin{itemize}
		\item Geschichte der KORE kennen lernen
		\item Aufgaben der KORE und Einordnung in das betriebliche Rechnungswesen verstehen
		\item Wertebenen im Rechnungswesen unterscheiden
	\end{itemize}
	
	Grundlagen der KORE
	\begin{itemize}
		\item Kostenwürfel als Hilfsmittel verwenden
		\item Überleitung von externem zu internem Rechnungswesen durchführen
	\end{itemize}
\end{frame}

\begin{frame}
	\frametitle{Titel der Folie\\maximal zwei Zeilen}
	Text Ebene 1
	\begin{itemize}
		\item Zweite Ebene
		\begin{itemize}
			\item Dritte Ebene
			\begin{itemize}
				\item Vierte Ebene
			\end{itemize}
		\end{itemize}
	\end{itemize}
\end{frame}

%%%%%%%%%%%%%%%%%%%%%%%%%%%%%%%%%%%%%%%%%%%%%%%%%%%%%%%%%%%%%%%%%%%%%%%%%%%%
\section{Aufzählungen}
%%%%%%%%%%%%%%%%%%%%%%%%%%%%%%%%%%%%%%%%%%%%%%%%%%%%%%%%%%%%%%%%%%%%%%%%%%%%

\section{Formeln und Links}
\begin{frame}
	\frametitle{Formeln und Links}
	Eine Formel:
	\begin{eqnarray*}
	u(x,t) & = & \sum_{k=1}^{\infty} f_k \sin \frac{k \pi x}{L} \cos 
				 \frac{k \pi t}{aL} + \\
		   & + & \sum_{k=1}^{\infty} g_k \sin \frac{k \pi x}{L} \sin 
				 \frac{k \pi t}{aL} \\
	\end{eqnarray*}

	Ein Link:
	\begin{center}
		\url{http://www.tugraz.at}		
	\end{center}
\end{frame}

%%%%%%%%%%%%%%%%%%%%%%%%%%%%%%%%%%%%%%%%%%%%%%%%%%%%%%%%%%%%%%%%%%%%%%%%%%%%
\section{Quelltexte}
\begin{frame}[fragile]
	\frametitle{Quelltexte}
	\begin{spacing}{1}
	\begin{semiverbatim}
SUCHE (A,x)
1: i = 0
2: WHILE i<n
3:     i = i+1
4:     \alert{IF A[i]=x THEN RETURN i}
5: ELSE RETURN -1
	\end{semiverbatim}
	\end{spacing}
\end{frame}

%%%%%%%%%%%%%%%%%%%%%%%%%%%%%%%%%%%%%%%%%%%%%%%%%%%%%%%%%%%%%%%%%%%%%%%%%%%%

\section{Spalten und Grafiken}
\begin{frame}
	\frametitle{Zwei Inhalte Links/Rechts\\Wahlweise Text/Grafik}
	\begin{columns}[onlytextwidth]
		\begin{column}{0.5\textwidth}
			\begin{itemize}
				\item Lorem ipsum dolor sit amet, consectetur 
				\item adipisicing elit, sed do eiusmod tempor 
				\item incididunt ut labore et dolore magna aliqua. 
				\item Ut enim ad minim veniam, quis nostrud 
			\end{itemize}
		\end{column}
		\begin{column}{0.5\textwidth}
			\begin{center}
			\includegraphics[width=0.5\textwidth]{logo.pdf}\\
			Grafik in Spalte 2
			\end{center}
		\end{column}
	\end{columns}
\end{frame}

\section{Weitere Beispielfolien}

\begin{frame}
	\frametitle{Nur Titel \\maximal 2-zeilig}
\end{frame}



%%%%%%%%%%%%%%%%%%%%%%%%%%%%%%%%%%%%%%%%%%%%%%%%%%%%%%%%%%%%%%%%%%%%%%%%%%%%
\section{Outlook}
%%%%%%%%%%%%%%%%%%/%%%%%%%%%%%%%%%%%%%%%%%%%%%%%%%%%%%%%%%%%%%%%%%%%%%%%%%%%%

\begin{frame}
	\frametitle{Outlook}
	\begin{itemize}
		\item Master Project
		\item Encryption and Signature Validation
		\item Better API
		\item Keyservers?
	\end{itemize}
	\begin{itemize}
		\item Provide Anonymity? (Metadata?)\footnotemark
		 \item Provide Integrity for Headers?
	\end{itemize}
	\footnotetext{http://grimoire.ca/gpg/terrible}
\end{frame}

\sectionheader[Questions? Remarks?]{Thank You for Your Attention}

%%%%%%%%%%%%%%%%%%%%%%%%%%%%%%%%%%%%%%%%%%%%%%%%%%%%%%%%%%%%%%%%%%%%%%%%%%%%
\end{document}
%%%%%%%%%%%%%%%%%%%%%%%%%%%%%%%%%%%%%%%%%%%%%%%%%%%%%%%%%%%%%%%%%%%%%%%%%%%%

%% EOF
