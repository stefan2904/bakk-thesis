% LaTeX Präsentationsvorlage (2013) der TU Graz, rev12, 2013/01/31
\documentclass{beamer}
% \documentclass[aspectratio=169]{beamer}
% \usetheme{tugraz2013}
% \usetheme[notes]{tugraz2013}
\usetheme[minimal, notes]{tugraz2013}

%% Titelblatt-Einstellungen
\title[openPGP in Java]{\tiny{Java Privacy Guard}\\\normalsize An Implementation of the \\ OpenPGP Message Format in Java}
\author{Stefan More}
% \date{Graz, XX. Dezember 2010}		% \today für heutiges Datum verwenden
\date{June 13, 2014}
\institute[Bakk@IAIK]{\\Advisor: Dieter Bratko}
%\instituteurl{iaik.tugraz.at}
\institutelogo{javaSec.jpg}
\additionallogo{institutslogo.png}

%%%%%%%%%%%%%%%%%%%%%%%%%%%%%%%%%%%%%%%%%%%%%%%%%%%%%%%%%%%%%%%%%%%%%%%%%%%%
\begin{document}
%%%%%%%%%%%%%%%%%%%%%%%%%%%%%%%%%%%%%%%%%%%%%%%%%%%%%%%%%%%%%%%%%%%%%%%%%%%%
\titleframe[This is a second-screen test. \\
		Hello, second-screen! \\
		\textbf{If audience can read this, something is wrong}.]

\begin{frame}
  \frametitle{Agenda}
  \tableofcontents%[hideallsubsections] 
  \note{
  	First I'm going to talk about what openPGP is \ldots \\ 
  	and do a short comparison to other protocols \ldots \\
  	last but not least i am going to introduce you to my implementation \ldots \\
  	and sum up what i have done
  }
\end{frame}

%%%%%%%%%%%%%%%%%%%%%%%%%%%%%%%%%%%%%%%%%%%%%%%%%%%%%%%%%%%%%%%%%%%%%%%%%%%%
\section{Problem Statement}
%%%%%%%%%%%%%%%%%%%%%%%%%%%%%%%%%%%%%%%%%%%%%%%%%%%%%%%%%%%%%%%%%%%%%%%%%%%%

{ % all template changes are local to this group.
    \setbeamertemplate{navigation symbols}{}
    \begin{frame}[plain]
        \begin{tikzpicture}[remember picture,overlay]
            \node[anchor=north west,inner sep=0pt] at (-1, 2) {
                \includegraphics[width=\paperwidth]{media/internet.png}
            };
        
        \pause
        
            \node[anchor=north west,inner sep=0pt] at (0, 0) {
                \includegraphics[width=65pt]{media/greenwald.png}
            };
        
            \node[anchor=north west,inner sep=0pt] at (0, -4) {
                \includegraphics[width=65pt]{media/poitras.jpg}
            };
       
            \node[anchor=north west,inner sep=0pt] at (8, -5) {
                \includegraphics[width=65pt]{media/snowden.jpg}
            };

        \pause
        
            \node[anchor=north west,inner sep=0pt] at (8, 1) {
                \includegraphics[width=60pt]{media/servers.png}
            };
        	 \draw [<->, line width=0.2cm, red, dashed] (9, -5) -- (9, -1.5); % line from server to snowden
        	 
        % \pause
        	 
        	 \draw [<->, line width=0.2cm, red, dashed] (2.5, -2) -- (8, 0); % line from server to greenwald
        	 \draw [<->, line width=0.2cm, red, dashed] (2.5, -6) -- (8, -1.5); % line from server to poitras
        	 
        	 \node[anchor=north west,inner sep=0pt] at (9, -3) {
                \includegraphics[width=30pt]{media/lock.png}
            };
            \node[anchor=north west,inner sep=0pt] at (4, -4) {
                \includegraphics[width=30pt]{media/lock.png}
            };

			\pause            
            
            \node[anchor=north west,inner sep=0pt] at (4, -0.5) {
                \includegraphics[width=30pt]{media/attack.png}
            };
            
             \node[anchor=north west,inner sep=0pt] at (9, 0) {
                \includegraphics[width=30pt]{media/attack.png}
            };
        	 
        \end{tikzpicture}
        
         \note {
     		Take the internet as an example. \\
			Well, lets say we have two persons who have good reasons \\
			to communicate confidentiality over such a network.		\\
			\ldots \\
			Laura POITRAS, glenn greenwald \\
			and of course edward snowden
			\ldots \\
			
			 \textbf{provided SSL / TLS is secure}
     }
     \end{frame}    
}

%%%%%%%%%%%%%%%%%%%%%%%%%%%%%%%%%%%%%%%%%%%%%%%%%%%%%%%%%%%%%%%%%%%%%%%%%%%%

\begin{frame}

% me trying to mail with customer -> konkurenz listening
% cloud in between? internet routing via england?

\end{frame}

%%%%%%%%%%%%%%%%%%%%%%%%%%%%%%%%%%%%%%%%%%%%%%%%%%%%%%%%%%%%%%%%%%%%%%%%%%%%

\begin{frame}

% hospital trying to mail with client -> employer listening
% gmail in between?

\end{frame}

%%%%%%%%%%%%%%%%%%%%%%%%%%%%%%%%%%%%%%%%%%%%%%%%%%%%%%%%%%%%%%%%%%%%%%%%%%%%

\begin{frame}

% slide which explains why we need privacy (even if we have nothing to hide?)
% explain why ssl is not enough => gmail == google, nsa has access, wirtschaftsspionage, etc

\end{frame}

% short extra slide?
% we need confidential, integrity, authentizity.
% end to end
% pgp provides that

%%%%%%%%%%%%%%%%%%%%%%%%%%%%%%%%%%%%%%%%%%%%%%%%%%%%%%%%%%%%%%%%%%%%%%%%%%%%
\section{openPGP}
%%%%%%%%%%%%%%%%%%%%%%%%%%%%%%%%%%%%%%%%%%%%%%%%%%%%%%%%%%%%%%%%%%%%%%%%%%%%


\begin{frame}
\note{
	 Zimmermann: anti-nuclear activist \\
	 PGP: invented to store message on message boards	
}
\frametitle{PGP History}
\begin{itemize}
	\item Pretty Good Privacy
	\item 1991: PGP created by Phil Zimmermann 
	\item \textit{most widely used email encryption software in the world}
	\item 1998: PGP 5 standardized: \textbf{OpenPGP (RFC 4880)}
	\item 2010: PGP assets sold to Symantec for \$ 300.000.000 (Enterprise Security Group)
\end{itemize}

\end{frame}

%%%%%%%%%%%%%%%%%%%%%%%%%%%%%%%%%%%%%%%%%%%%%%%%%%%%%%%%%%%%%%%%%%%%%%%%%%%%
\begin{frame}

% explain what gpg is, what pgp is, what openpgp is and what jog is (bouncycastle, too)

	\begin{itemize}
		\item Standard: openPGP (RFC 4880 et al.)
		\item Implementations:
		\begin{itemize}
		\item PGP (Zimmermann, Symantec)
		\item GnuPG / GPG (GNU)
		\item APG (Android)
		\item End-to-end (Google, Chrome)
		\item Bouncycastle (Java)
		\item \textit{Java Privacy Guard} (IAIK-JCE)
		\end{itemize}
	\end{itemize}

\end{frame}

%%%%%%%%%%%%%%%%%%%%%%%%%%%%%%%%%%%%%%%%%%%%%%%%%%%%%%%%%%%%%%%%%%%%%%%%%%%%
\begin{frame}

% again, slide with snowden & greenwald + cloud + secret services

\end{frame}

%%%%%%%%%%%%%%%%%%%%%%%%%%%%%%%%%%%%%%%%%%%%%%%%%%%%%%%%%%%%%%%%%%%%%%%%%%%%

\begin{frame}

% adding gpg: 
% snowden knows its greenwald
% snowden know message not changed
% nsa does not know what they write
% BUT: nsa knows (possibly) THAT they write -> TOR solves that?

\end{frame}

%%%%%%%%%%%%%%%%%%%%%%%%%%%%%%%%%%%%%%%%%%%%%%%%%%%%%%%%%%%%%%%%%%%%%%%%%%%%

\begin{frame}

% summing up, pgp provides confidential, integrity, authentizity
% but not anonymity (metadata)

\end{frame}

%%%%%%%%%%%%%%%%%%%%%%%%%%%%%%%%%%%%%%%%%%%%%%%%%%%%%%%%%%%%%%%%%%%%%%%%%%%%

\begin{frame}

% how does pgp do that? -> explain encryption (confidentiality)

\end{frame}

%%%%%%%%%%%%%%%%%%%%%%%%%%%%%%%%%%%%%%%%%%%%%%%%%%%%%%%%%%%%%%%%%%%%%%%%%%%%

\begin{frame}

% how does pgp do that? -> explain message signing (integrity)

\end{frame}

%%%%%%%%%%%%%%%%%%%%%%%%%%%%%%%%%%%%%%%%%%%%%%%%%%%%%%%%%%%%%%%%%%%%%%%%%%%%

\begin{frame}

% so... pgp == s/mime ???

\end{frame}

%%%%%%%%%%%%%%%%%%%%%%%%%%%%%%%%%%%%%%%%%%%%%%%%%%%%%%%%%%%%%%%%%%%%%%%%%%%%

\begin{frame}

% NO! => explain authentizy => web of trust
% advantages: does not need ca, everyone is ca
% dissadvantages: social graph public (again: no anonymity!), how secure is a ca? usability?
% concusio: bussines vs private? central vs decentral

\end{frame}

%%%%%%%%%%%%%%%%%%%%%%%%%%%%%%%%%%%%%%%%%%%%%%%%%%%%%%%%%%%%%%%%%%%%%%%%%%%%


\begin{frame}
	\frametitle{Ziele der ersten Vorlesung}
	Einführung in die KORE
	\begin{itemize}
		\item Geschichte der KORE kennen lernen
		\item Aufgaben der KORE und Einordnung in das betriebliche Rechnungswesen verstehen
		\item Wertebenen im Rechnungswesen unterscheiden
	\end{itemize}
	
	Grundlagen der KORE
	\begin{itemize}
		\item Kostenwürfel als Hilfsmittel verwenden
		\item Überleitung von externem zu internem Rechnungswesen durchführen
	\end{itemize}
\end{frame}

\begin{frame}
	\frametitle{Titel der Folie\\maximal zwei Zeilen}
	Text Ebene 1
	\begin{itemize}
		\item Zweite Ebene
		\begin{itemize}
			\item Dritte Ebene
			\begin{itemize}
				\item Vierte Ebene
			\end{itemize}
		\end{itemize}
	\end{itemize}
\end{frame}

%%%%%%%%%%%%%%%%%%%%%%%%%%%%%%%%%%%%%%%%%%%%%%%%%%%%%%%%%%%%%%%%%%%%%%%%%%%%
\section{Related Protocols}
%%%%%%%%%%%%%%%%%%%%%%%%%%%%%%%%%%%%%%%%%%%%%%%%%%%%%%%%%%%%%%%%%%%%%%%%%%%%

%% Evtl streichen weil sosnt zu viel?
% vergleich mit smime shcon im openpgp kapitel,
% vergleich mit ssl ganz oben?

%%%%%%%%%%%%%%%%%%%%%%%%%%%%%%%%%%%%%%%%%%%%%%%%%%%%%%%%%%%%%%%%%%%%%%%%%%%%
\section{Java Privacy Guard}
%%%%%%%%%%%%%%%%%%%%%%%%%%%%%%%%%%%%%%%%%%%%%%%%%%%%%%%%%%%%%%%%%%%%%%%%%%%%

\begin{frame}[fragile]
	\frametitle{Quelltexte}
	\begin{spacing}{1}
	\begin{semiverbatim}
SUCHE (A,x)
1: i = 0
2: WHILE i<n
3:     i = i+1
4:     \alert{IF A[i]=x THEN RETURN i}
5: ELSE RETURN -1
	\end{semiverbatim}
	\end{spacing}
\end{frame}

%%%%%%%%%%%%%%%%%%%%%%%%%%%%%%%%%%%%%%%%%%%%%%%%%%%%%%%%%%%%%%%%%%%%%%%%%%%%


\begin{frame}
	\frametitle{Zwei Inhalte Links/Rechts\\Wahlweise Text/Grafik}
	\begin{columns}[onlytextwidth]
		\begin{column}{0.5\textwidth}
			\begin{itemize}
				\item Lorem ipsum dolor sit amet, consectetur 
				\item adipisicing elit, sed do eiusmod tempor 
				\item incididunt ut labore et dolore magna aliqua. 
				\item Ut enim ad minim veniam, quis nostrud 
			\end{itemize}
		\end{column}
		\begin{column}{0.5\textwidth}
			\begin{center}
			\includegraphics[width=0.5\textwidth]{logo.pdf}\\
			Grafik in Spalte 2
			\end{center}
		\end{column}
	\end{columns}
\end{frame}



\begin{frame}
	\frametitle{Nur Titel \\maximal 2-zeilig}
\end{frame}



%%%%%%%%%%%%%%%%%%%%%%%%%%%%%%%%%%%%%%%%%%%%%%%%%%%%%%%%%%%%%%%%%%%%%%%%%%%%
\section{Summary \&{} Outlook}
%%%%%%%%%%%%%%%%%%%%%%%%%%%%%%%%%%%%%%%%%%%%%%%%%%%%%%%%%%%%%%%%%%%%%%%%%%%%

\begin{frame}
	\frametitle{Summary}
	
	\begin{itemize}
		\item Studied RFC 4880
		\item Studied IAIK-JCE
		\item Implemented:
		\begin{itemize}
			\item Reading ASCII-Armor (Base 64 + CRC24)
			\item Parsing PGP Message Objects (\textit{Packets})
			\item Key Management (Simple Keychain)
			\item Message Decryption (RSA \&{} ElGamal)
			\item Signature Validation (RSA \&{} DSA)
			\item Decompression
		\end{itemize}
	\end{itemize}
	
\end{frame}

%%%%%%%%%%%%%%%%%%%%%%%%%%%%%%%%%%%%%%%%%%%%%%%%%%%%%%%%%%%%%%%%%%%%%%%%%%%%

\begin{frame}
	\frametitle{Future (Research) Outlook}
	
	\begin{itemize}
		%\item Master Project
		\item Message Encryption and Signature Validation
		\item Missing parts of RFC 4880 (Partial body length, \ldots)
		\item RFC 6637: ECC
		\item Keyservers?
	\end{itemize}
	Out of scope:
	\begin{itemize}
	    \item RFC 3156: PGP/MIME
		\item Provide Anonymity? (Metadata!)%\footnotemark
		\item Provide Integrity for Mail-Headers?
		\item Protocol/Implementation Usability?
	\end{itemize}
	%\footnotetext{http://grimoire.ca/gpg/terrible}
	
	\note{
		 Master Project, Summer job, \\
		 further research?	
	}
\end{frame}

%%%%%%%%%%%%%%%%%%%%%%%%%%%%%%%%%%%%%%%%%%%%%%%%%%%%%%%%%%%%%%%%%%%%%%%%%%%%

\begin{frame}
\frametitle{Sources}
\small
\begin{thebibliography}{1}
	\bibitem{snowden} Laura Poitras \small \url{https://commons.wikimedia.org/wiki/File:Edward_Snowden-2.jpg} 
	
	\bibitem{greenwald} Glenn Greenwald \small \url{https://en.wikipedia.org/wiki/File:Glenn_greenwald_portrait_transparent.png}
	
	\bibitem{poitras} Katy Scoggin \small \url{https://commons.wikimedia.org/wiki/File:Laura_Poitras_2014.jpg}
	
	\bibitem{servers} RRZE \item \url{https://commons.wikimedia.org/wiki/Category:RRZE-Icon-Set}
	
	\bibitem{internet}  The Opte Project \small \url{http://www.opte.org/the-internet}

\end{thebibliography}
\end{frame}

\sectionheader[Questions? Remarks?]{Thank You for Your Attention}

%%%%%%%%%%%%%%%%%%%%%%%%%%%%%%%%%%%%%%%%%%%%%%%%%%%%%%%%%%%%%%%%%%%%%%%%%%%%
\end{document}
%%%%%%%%%%%%%%%%%%%%%%%%%%%%%%%%%%%%%%%%%%%%%%%%%%%%%%%%%%%%%%%%%%%%%%%%%%%%

%% EOF
