\chapter{The \textit{Java Privacy Guard}} \label{chapter:jpg}

% explain what was done in java (and what not?)

As practical part of this thesis we did a implementation of the openPGP message format \citep{RFC4880} in the \textit{Java} programing language, creating a library called \textit{Java Privacy Guard}.

For cryptographic primitives like basic data types and encryption and digest algorithms we used the \textit{IAIK-JCE} library. \\

%TODO: figure with (clean) class-diagram of (important) JPG classes?

%\section{Introduction}


% only discuss how things mentioned in chapter 3 and 4 (the packets/transferables) are implemented in java

The \textit{Packet}- and \textit{Transferable}-structure as described in chapter \ref{chapter:messageformat} represents the base-structure of our implementation. All \textit{Packets} defined in \citep[section 5]{RFC4880} are implemented as one or more \textit{Java}-classes. Additionally all relevant \textit{Packet}-compositions are implemented in the form of \textit{Transferable}-classes as defined in \citep[section 11]{RFC4880}.

\section{The APIs}

% explain difference between high-/low-level api

To facilitate the use of our implementation we provide \textbf{two APIs} for access to the library functionality. \\  

The \textit{high-level API} represents the interface used by users of our library. It mostly consists of \textit{Transferable}-classes and related toolbox-classes.

The \textit{low-level API} represents the inner working of the library. It is made up of all \textit{Packet}-classes and toolbox-classes needed to work with them. For the sake of usability (and extensibility in the future) this classes are hidden from library-users. \\

To the time of writing of this thesis, it was not clear in detail which part of the library belongs to which API. 

\section{The \textit{PGPObjectStream}}



%\section{Issues during Implementation}
%
%\subsection{Unsigned and Java}
%
%\subsection{RFC}
%
%--- Ab hier section Evaluation? ---
%
%\section{Interoperability}
%
%\subsection{Bouncycastle}
%
%\subsection{GnuPG}

\section{Implementation Security}

Security concerns, as long as they affect the actual design of the openPGP standard, are discussed in chapter \ref{chapter:concerns} and in \citep[section 14]{RFC4880}. In the following section, we give an overview on how they effect our implementation. 
% We are closing this chapter with a short overview of countermeasures we implemented in our implementation.

\subsection{Possible Attacks}


%\subsection{Countermeasures}


%\section{Status of implementation (?)}






