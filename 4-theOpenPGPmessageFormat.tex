\chapter{The openPGP message format}

% explain its inner workings, what a packet is and how packets work

% explain that all is binary and what ascii armor is (ignore crc24 checksum?)

% explain how the principles from chapter 3 are achieved usign this packet structures

% explain what is encrypted
% explain what is signed in case of message signing
% explain what is signed in case of key certificating

% explain things like cleartext signatures and detached signatures

% ignore the following section marks (replace them with better ones!)
% instead, list commons tasks and explain how this is achived (diagramm?) and what packets are used (make tables for interesting packets)

% interesting things: transferables (key, encrypted message+signature)

\section{Principles}

%\section{Data primitives}

\section{Keys}

\section{Messages}

% \subsection{Multipart}

\section{Signatures}

\section{Compression}

\section{Algorithms and Parameters}

% list algorithms from rfc, give short overview
% mention that seciruty stuff is in last chapter

% also mention IANA specification (https://www.iana.org/assignments/pgp-parameters/pgp-parameters.xhtml ?

The openPGP standard allows the usage of various algorithms for encryption, integrity protection and signing. The following section lists the algorithms specified in \citep[section 9]{RFC4880}. A brief summary of those algorithm regarding their state of security and recommended key lengths is given in chapter \ref{chapter:concerns}.

\subsection{Public-Key Algorithms}

\subsection{Symmetric-Key Algorithms}

\subsection{Hash Algorithms}

