\chapter{The openPGP message format} \label{chapter:messageformat}

% explain its inner workings, what a packet is and how packets work

% explain that all is binary and what ascii armor is (ignore crc24 checksum?)

% explain how the principles from chapter 3 are achieved usign this packet structures

% explain what is encrypted
% explain what is signed in case of message signing
% explain what is signed in case of key certificating

% explain things like cleartext signatures and detached signatures

%  list commons tasks and explain how this is achived (diagramm?) and what packets are used (make tables for interesting packets)

% interesting things: transferables (key, encrypted message+signature)

% \section{Structure}

In this chapter we explain the principles behind the inner working of openPGP. 
Furthermore we illustrate some important data structures defined by openPGP.  

We show how openPGP carries out the principles described in chapter \ref{chapter:openpgp} and give an overview of the algorithms used in openPGP.

\subsection{Packets}

The basic building blocks of openPGP are called \textit{Packets} \citep[section 4]{RFC4880}. \textit{Packets} are a structured and defined sequence of bytes.

Each \textit{Packet} consists of a header and a body. The header describes the type of the \textit{Packet} and its length. The body contains the actual \textit{Packet}.

OpenPGP defines two different versions of \textit{Packets}. They main difference between the version is the structure of the packet-header. The new version also allows 64 different \textit{Packet} types. The old version reserves only 4 bit for the \textit{Packet} type and therefore allows a maximum of 16 different \textit{Packet} types. \\

OpenPGP currently defines 17 different \textit{Packet} types. \todo{table} gives an overview of the supported \textit{Packet} types. A detailed description can be found in \citep[section 5]{RFC4880}.

To illustrate the principles described here, \todo{} shows the structure of a \textit{key packet} as an example.

\todo{example packet, key?}

\todo{more examples? a message? a signature with subpacket?}

\subsection{Transferables}

\textit{Packets} are representations of very basic data structures. To create openPGP messages and keys some of this \textit{Packets} are composed in a meaningful way \citep[section 11]{RFC4880}. A (defined) composition of \textit{Packets} is called \textit{Transferable}, since its purpose is to be transfered from one openPGP implementation to another or just to be stored and used later. 

openPGP defines four different \textit{Transferable} types \citep[section 11]{RFC4880}. \\

First of all it is necessary to export public and secret keys in form of \textbf{Transferable Public Keys} and \textbf{Transferable Private Keys}. This enables storing the keys in \textit{keyrings} while the openPGP implementation is halted.  Additionally it is necessary to transmit the public key to other parties of the communication, as described in section \ref{section:openpgp:keydistributio}. \\

\todo{table: packet structure of a pgp key with some signatures?}

The core functionality of openPGP is to securely transmit messages. The third type of \textit{Transferables} therefore is a composition of \textit{Packets} representing such a \textbf{OpenPGP Message}. \\

\todo{table: packet structure of a pgp message (like on the whiteboard?)}

Furthermore it is possible to transmit signatures separated from the actual signed message, so called \textbf{Detached Signatures}.

\subsection{Representation}

Since openPGP defines the structure of \textit{Packets} as a sequence of arbitrary bytes, the primary representation of a \textit{Transferable} is also in byte format.

For many use-cases it is sufficient to store and transmit a \textit{Transferable} in binary form.

Additionally it is possible to convert a binary \textit{Transferable} to ASCII-format for sending it  via channels which can not deal with binary data. Such an representation of a binary openPGP \textit{Transferable} is called \textit{ASCII-armored}.

\todo{sample ascii armored stuff (a mesage and a key?)}

\section{Keys}

\section{Messages}

% encryption and signing

% \subsection{Compression}

% \subsection{Multipart}

\section{Key Certification}


\subsection{Key Revocation}
%\subsection{Revocation Signatures}


\section{Algorithms and Parameters}

% list algorithms from rfc, give short overview
% mention that seciruty stuff is in last chapter

% also mention IANA specification (https://www.iana.org/assignments/pgp-parameters/pgp-parameters.xhtml ?

The openPGP standard allows the usage of various algorithms for encryption, integrity protection and signing. The following section lists the algorithms specified in \citep[section 9]{RFC4880}. A brief summary of those algorithm regarding their state of security and recommended key lengths is given in chapter \ref{chapter:concerns}.

\subsection{Public-Key Algorithms}

\subsection{Symmetric-Key Algorithms}

\subsection{Hash Algorithms}

